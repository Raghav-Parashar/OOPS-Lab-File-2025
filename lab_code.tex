\documentclass[a4paper,12pt]{article}
\usepackage[utf8]{inputenc}
\usepackage{geometry}
\geometry{margin=1in}
\usepackage{listings}
\usepackage{xcolor}
\usepackage{fancyhdr}
\usepackage{lastpage}

% Fix fancyhdr headheight warning
\setlength{\headheight}{14.5pt}

% Define Java syntax highlighting for code
\lstset{
    language=Java,
    basicstyle=\ttfamily\scriptsize,
    keywordstyle=\color{blue}\bfseries,
    stringstyle=\color{red},
    commentstyle=\color{gray}\itshape,
    numbers=left,
    numberstyle=\tiny,
    stepnumber=1,
    numbersep=5pt,
    showspaces=false,
    showstringspaces=false,
    frame=single,
    breaklines=true,
    breakatwhitespace=true,
    tabsize=4,
    captionpos=b
}

% Define output styling using listings (instead of verbatim)
\lstnewenvironment{outputlisting}{%
    \lstset{
        basicstyle=\ttfamily\scriptsize,
        numbers=none,
        frame=single,
        breaklines=true,
        breakatwhitespace=true,
        showspaces=false,
        showstringspaces=false,
        backgroundcolor=\color{lightgray}
    }%
}{}

% Header and Footer
\pagestyle{fancy}
\fancyhf{}
\fancyhead[L]{OOPS Lab File 2025}
\fancyhead[R]{Woxsen University}
\fancyfoot[C]{Page \thepage\ of \pageref{LastPage}}

% Title
\title{\textbf{OOPS Lab File 2025} \\ School of Technology, Woxsen University}
\author{}
\date{March 23, 2025}

\begin{document}

\maketitle
\tableofcontents
\newpage

\section{Experiment 1: Java Basic Programming}

\clearpage
\subsection{1.2: Evaluate Specified Expressions}
\begin{samepage}
\begin{lstlisting}[caption={ExpressionEvaluator.java}]
public class ExpressionEvaluator {
    public static void main(String[] args) {
        System.out.println("Expression 1: (101 + 0) / 3");
        double result1 = (101 + 0) / 3.0;
        System.out.println("Result 1: " + result1);
        
        System.out.println("\nExpression 2: (3.0e-6 * 10000000.1)");
        double result2 = 3.0e-6 * 10000000.1;
        System.out.println("Result 2: " + result2);
        
        System.out.println("\nExpression 3: (true && true)");
        boolean result3 = true && true;
        System.out.println("Result 3: " + result3);
        
        System.out.println("\nExpression 4: (false && true)");
        boolean result4 = false && true;
        System.out.println("Result 4: " + result4);
        
        System.out.println("\nExpression 5: (false && false) || (true && true)");
        boolean result5 = (false && false) || (true && true);
        System.out.println("Result 5: " + result5);
        
        System.out.println("\nExpression 6: (false || false) && (true && true)");
        boolean result6 = (false || false) && (true && true);
        System.out.println("Result 6: " + result6);
    }
}
\end{lstlisting}

\begin{outputlisting}
Expression 1: (101 + 0) / 3
Result 1: 33.666666666666664

Expression 2: (3.0e-6 * 10000000.1)
Result 2: 30.0000003

Expression 3: (true && true)
Result 3: true

Expression 4: (false && true)
Result 4: false

Expression 5: (false && false) || (true && true)
Result 5: true

Expression 6: (false || false) && (true && true)
Result 6: false
\end{outputlisting}
\end{samepage}

\clearpage
\subsection{1.3: Fahrenheit to Celsius Converter}
\begin{samepage}
\begin{lstlisting}[caption={TempConverter.java}]
import java.util.Scanner;

public class TempConverter {
    public static void main(String[] args) {
        Scanner sc = new Scanner(System.in);
        
        System.out.print("Input a degree in Fahrenheit: ");
        double fahrenheit = sc.nextDouble();
        
        double celsius = (fahrenheit - 32) * 5.0 / 9.0;
        System.out.println(fahrenheit + " degree Fahrenheit is equal to " + celsius + " in Celsius");
        
        sc.close();
    }
}
\end{lstlisting}

\begin{outputlisting}
Input a degree in Fahrenheit: 212
212.0 degree Fahrenheit is equal to 100.0 in Celsius
\end{outputlisting}
\end{samepage}

\section{Experiment 2: Conditional Statement}

\clearpage
\subsection{2.1: Check Four Integers Equal}
\begin{samepage}
\begin{lstlisting}[caption={CheckEqual.java}]
import java.util.Scanner;

public class CheckEqual {
    public static void main(String[] args) {
        Scanner sc = new Scanner(System.in);
        System.out.println("Enter four integers:");
        int a = sc.nextInt();
        int b = sc.nextInt();
        int c = sc.nextInt();
        int d = sc.nextInt();
        
        if (a == b && b == c && c == d) {
            System.out.println("equal");
        } else {
            System.out.println("not equal");
        }
        
        sc.close();
    }
}
\end{lstlisting}

\begin{outputlisting}
Enter four integers:
5 5 5 5
equal
\end{outputlisting}
\end{samepage}

\clearpage
\subsection{2.2: Check Range of Two Doubles}
\begin{samepage}
\begin{lstlisting}[caption={CheckRange.java}]
import java.util.Scanner;

public class CheckRange {
    public static void main(String[] args) {
        Scanner sc = new Scanner(System.in);
        
        System.out.println("Enter two double values:");
        double num1 = sc.nextDouble();
        double num2 = sc.nextDouble();
        
        boolean result = (num1 > 0 && num1 < 1) && (num2 > 0 && num2 < 1);
        System.out.println("Both numbers are strictly between 0 and 1: " + result);
        
        sc.close();
    }
}
\end{lstlisting}

\begin{outputlisting}
Enter two double values:
0.5 0.7
Both numbers are strictly between 0 and 1: true
\end{outputlisting}
\end{samepage}

\section{Experiment 3: Introduction to Arrays in Java}

\clearpage
\subsection{3.1: Print 2D Boolean Array}
\begin{samepage}
\begin{lstlisting}[caption={BooleanArrayPrinter.java}]
public class BooleanArrayPrinter {
    public static void main(String[] args) {
        boolean[][] array = {{true, false, true}, {false, true, false}};
        
        for (int i = 0; i < array.length; i++) {
            for (int j = 0; j < array[i].length; j++) {
                System.out.print(array[i][j] ? "t " : "f ");
            }
            System.out.println();
        }
    }
}
\end{lstlisting}

\begin{outputlisting}
t f t 
f t f 
\end{outputlisting}
\end{samepage}

\clearpage
\subsection{3.2: Transpose 2D Array}
\begin{samepage}
\begin{lstlisting}[caption={ArrayTranspose.java}]
public class ArrayTranspose {
    public static void main(String[] args) {
        int[][] original = {{10, 20, 30}, {40, 50, 60}};
        int[][] transposed = new int[3][2];
        
        for (int i = 0; i < original.length; i++) {
            for (int j = 0; j < original[i].length; j++) {
                transposed[j][i] = original[i][j];
            }
        }
        
        for (int i = 0; i < transposed.length; i++) {
            for (int j = 0; j < transposed[i].length; j++) {
                System.out.print(transposed[i][j] + " ");
            }
            System.out.println();
        }
    }
}
\end{lstlisting}

\begin{outputlisting}
10 40 
20 50 
30 60 
\end{outputlisting}
\end{samepage}

\section{Experiment 4: 2-Dimensional Array in Java}

\clearpage
\subsection{4.1: Prime Number Condition Array}
\begin{samepage}
\begin{lstlisting}[caption={PrimeArray.java}]
public class PrimeArray {
    public static boolean isPrime(int n) {
        if (n < 2) return false;
        for (int i = 2; i <= Math.sqrt(n); i++) {
            if (n % i == 0) return false;
        }
        return true;
    }
    
    public static void main(String[] args) {
        int m = 5;
        boolean[][] A = new boolean[m][m];
        
        for (int i = 0; i < m; i++) {
            for (int j = 0; j < m; j++) {
                A[i][j] = !(isPrime(i) && isPrime(j));
                System.out.print((A[i][j] ? "t " : "f ") + " ");
            }
            System.out.println();
        }
    }
}
\end{lstlisting}

\begin{outputlisting}
t t t t t 
t t t t t 
t t f f t 
t t f f t 
t t t t t 
\end{outputlisting}
\end{samepage}

\clearpage
\subsection{4.2: K Largest Elements}
\begin{samepage}
\begin{lstlisting}[caption={KLargest.java}]
import java.util.Arrays;
import java.util.Collections;

public class KLargest {
    public static void main(String[] args) {
        Integer[] array = {4, 2, 9, 7, 5, 6, 1, 3};
        int k = 3;
        
        Arrays.sort(array, Collections.reverseOrder());
        
        System.out.println("The " + k + " largest elements are:");
        for (int i = 0; i < k; i++) {
            System.out.print(array[i] + " ");
        }
    }
}
\end{lstlisting}

\begin{outputlisting}
The 3 largest elements are:
9 7 6 
\end{outputlisting}
\end{samepage}

\section{Experiment 5: Introduction of Class in Java}

\clearpage
\subsection{5.1: Vehicle and Car Classes}
\begin{samepage}
\begin{lstlisting}[caption={VehicleTest.java}]
class Vehicle {
    public void drive() {
        System.out.println("Driving a vehicle");
    }
}

class Car extends Vehicle {
    @Override
    public void drive() {
        System.out.println("Repairing a car");
    }
}

public class VehicleTest {
    public static void main(String[] args) {
        Car car = new Car();
        car.drive();
    }
}
\end{lstlisting}

\begin{outputlisting}
Repairing a car
\end{outputlisting}
\end{samepage}

\clearpage
\subsection{5.2: Shape and Rectangle Classes}
\begin{samepage}
\begin{lstlisting}[caption={ShapeTest.java}]
class Shape {
    public double getArea() {
        return 0.0;
    }
}

class Rectangle extends Shape {
    private double length;
    private double width;
    
    public Rectangle(double length, double width) {
        this.length = length;
        this.width = width;
    }
    
    @Override
    public double getArea() {
        return length * width;
    }
}

public class ShapeTest {
    public static void main(String[] args) {
        Rectangle rect = new Rectangle(5, 3);
        System.out.println("Rectangle Area: " + rect.getArea());
    }
}
\end{lstlisting}

\begin{outputlisting}
Rectangle Area: 15.0
\end{outputlisting}
\end{samepage}

\clearpage
\subsection{5.3: Employee and HRManager Classes}
\begin{samepage}
\begin{lstlisting}[caption={EmployeeTest.java}]
class Employee {
    public void work() {
        System.out.println("Employee is working");
    }
    public double getSalary() {
        return 50000.0;
    }
}

class HRManager extends Employee {
    @Override
    public void work() {
        System.out.println("HR Manager is managing employees");
    }
    
    public void addEmployee() {
        System.out.println("Adding a new employee");
    }
}

public class EmployeeTest {
    public static void main(String[] args) {
        HRManager hr = new HRManager();
        hr.work();
        System.out.println("Salary: " + hr.getSalary());
        hr.addEmployee();
    }
}
\end{lstlisting}

\begin{outputlisting}
HR Manager is managing employees
Salary: 50000.0
Adding a new employee
\end{outputlisting}
\end{samepage}

\section{Experiment 6: Java Class with Real-life Applications}

\clearpage
\subsection{6.1: BankAccount and SavingsAccount}
\begin{samepage}
\begin{lstlisting}[caption={BankTest.java}]
class BankAccount {
    protected double balance;
    
    public BankAccount() {
        this.balance = 0;
    }
    
    public void deposit(double amount) {
        balance += amount;
        System.out.println("Deposited: " + amount + ", New Balance: " + balance);
    }
    
    public void withdraw(double amount) {
        if (balance >= amount) {
            balance -= amount;
            System.out.println("Withdrawn: " + amount + ", New Balance: " + balance);
        } else {
            System.out.println("Insufficient funds");
        }
    }
}

class SavingsAccount extends BankAccount {
    @Override
    public void withdraw(double amount) {
        if (balance - amount < 100) {
            System.out.println("Cannot withdraw: Minimum balance of 100 required");
        } else {
            super.withdraw(amount);
        }
    }
}

public class BankTest {
    public static void main(String[] args) {
        SavingsAccount account = new SavingsAccount();
        account.deposit(500);
        account.withdraw(300);
        account.withdraw(150);
    }
}
\end{lstlisting}

\begin{outputlisting}
Deposited: 500.0, New Balance: 500.0
Withdrawn: 300.0, New Balance: 200.0
Cannot withdraw: Minimum balance of 100 required
\end{outputlisting}
\end{samepage}

\clearpage
\subsection{6.2: K Largest Elements}
\begin{samepage}
\begin{lstlisting}[caption={KLargest.java}]
import java.util.Arrays;
import java.util.Collections;

public class KLargest {
    public static void main(String[] args) {
        Integer[] array = {4, 2, 9, 7, 5, 6, 1, 3};
        int k = 3;
        
        Arrays.sort(array, Collections.reverseOrder());
        
        System.out.println("The " + k + " largest elements are:");
        for (int i = 0; i < k; i++) {
            System.out.print(array[i] + " ");
        }
    }
}
\end{lstlisting}

\begin{outputlisting}
The 3 largest elements are:
9 7 6 
\end{outputlisting}
\end{samepage}

\clearpage
\subsection{6.3: Vehicle Class Hierarchy}
\begin{samepage}
\begin{lstlisting}[caption={VehicleHierarchy.java}]
class Vehicle {
    protected String make, model;
    protected int year;
    protected String fuelType;
    
    public Vehicle(String make, String model, int year, String fuelType) {
        this.make = make;
        this.model = model;
        this.year = year;
        this.fuelType = fuelType;
    }
    
    public double calcFuelEfficiency() { return 0.0; }
    public double calcDistanceTravelled(double time, double speed) { return time * speed; }
    public double getMaxSpeed() { return 0.0; }
}

class Truck extends Vehicle {
    public Truck(String make, String model, int year, String fuelType) {
        super(make, model, year, fuelType);
    }
    @Override
    public double calcFuelEfficiency() { return 10.5; }
    @Override
    public double getMaxSpeed() { return 80.0; }
}

public class VehicleHierarchy {
    public static void main(String[] args) {
        Truck truck = new Truck("Ford", "F150", 2020, "Diesel");
        System.out.println("Truck Fuel Efficiency: " + truck.calcFuelEfficiency());
        System.out.println("Distance Travelled: " + truck.calcDistanceTravelled(2, 60));
        System.out.println("Max Speed: " + truck.getMaxSpeed());
    }
}
\end{lstlisting}

\begin{outputlisting}
Truck Fuel Efficiency: 10.5
Distance Travelled: 120.0
Max Speed: 80.0
\end{outputlisting}
\end{samepage}

\end{document}